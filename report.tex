\documentclass[a4paper,11pt]{article}
\usepackage[T1]{fontenc}
\usepackage[utf8]{inputenc}
\usepackage{lmodern}
\usepackage[francais]{babel}
\usepackage{geometry}

\geometry{hmargin=2.5cm,vmargin=2.5cm}

\title{Logique floue\\\texttt{Sauvons le RMS Titanic}}

\author{Alexis \textsc{Decker-Wurst}, Erwan  \textsc{Duroux}, Guillaume  \textsc{Laroyenne}}

\begin{document}

    \maketitle

    \begin{abstract}
        Ce document présente une synthèse du projet de logique floue que nous avons réalisé en deuxième année.\\
        Le cœur de ce projet est une bibliothèque de logique floue que nous avons programmé en \textit{C++}. La bibliothèque comporte les principaux opérateurs de bases de la logique floue ainsi que quelques méthodes de défuzzification tel que \textit{COQ}.\\
        Pour les utilisateurs puissent manipuler plus simplement cette bibliothèque, nous avons décidé d'ajouter un interpréteur permettant de modéliser des problèmes de logique floue sans avoir à la manipuler directement.\\
        La logique floue permettant de prendre des décisions complexe à partir d'un ensemble de données. Nous avons décidé de l'utiliser afin de réaliser un pilote automatique évitant les obstacles pouvant se trouver devant un navire, tout en utilisant la bibliothèque que nous avons conçu.
    \end{abstract}

    \section{Construction de la bibliothèque de logique floue}

    \section{Mise en place d'un interpréteur de logique floue}

    \section{Création du simulateur naval}

    \section{Modélisation du pilote automatique en logique floue}

\end{document}
