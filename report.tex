\documentclass[a4paper,11pt]{article}
\usepackage[T1]{fontenc}
\usepackage[utf8]{inputenc}
\usepackage{lmodern}
\usepackage[francais]{babel}
\usepackage{geometry}
\usepackage{float}
\usepackage{array}
\usepackage{multirow}
\usepackage{makecell}
\usepackage{graphicx}
\usepackage{listings}


\newcolumntype{M}[1]{>{\raggedright}m{#1}}

\geometry{hmargin=2.5cm,vmargin=2.5cm}

\title{Logique floue\\\texttt{Sauvons le RMS Titanic}}

\author{Alexis \textsc{Decker-Wurst}, Erwan  \textsc{Duroux}, Guillaume  \textsc{Laroyenne}}

\begin{document}

    \maketitle

    \begin{abstract}
        Ce document présente une synthèse du projet de logique floue que nous avons réalisé en deuxième année.\\
        Le cœur de ce projet est une bibliothèque de logique floue que nous avons programmé en \textit{C++}. La bibliothèque comporte les principaux opérateurs de bases de la logique floue ainsi que quelques méthodes de défuzzification.\\
        Pour les utilisateurs puissent manipuler plus simplement cette bibliothèque, nous avons décidé d'ajouter un interpréteur permettant de modéliser des problèmes de logique floue sans avoir à la manipuler directement.\\
        La logique floue permettant de prendre des décisions complexe à partir d'un ensemble de données. Nous avons décidé de l'utiliser afin de réaliser un pilote automatique évitant les obstacles pouvant se trouver devant un navire, tout en utilisant la bibliothèque que nous avons conçu.
    \end{abstract}

    \section{Construction de la bibliothèque de logique floue}

    \begin{figure}
        \begin{center}
            \caption{UML des packages du \texttt{Framework}}
            \includegraphics[scale=0.5]{assets/Packages_(UML).jpg}
            \label{fig:umlPackage}
        \end{center}
    \end{figure}


    \begin{table}
        \caption{Liste des opérateurs est opérandes du \textit{Framework}}
        \label{tab:listing}

        \begin{center}
            \begin{tabular}{|c|c|M{4cm}|}
                \hline
                Nom & Type & Description \tabularnewline
                \hline
                court texte & truc & Texte plus long qui sera centré dans la ligne et aligné à gauche dans la colonne  \tabularnewline
                \hline
            \end{tabular}
        \end{center}
    \end{table}
    
    \section{Mise en place d'un interpréteur de logique floue}

    \begin{figure}
        \begin{center}
            \caption{Exemple de code de l’interpréteur pour l’évaluation d'un pourboire}
            \lstinputlisting[language=Bash, firstline=0, lastline=9, frame=single]{assets/leaveatip_cog.fuzzy}
            \label{fig:codeExemple}
        \end{center}
    \end{figure}

    \section{Création du simulateur naval}

    \begin{figure}
        \begin{center}
            \caption{Schéma présentant les forces physique appliquées sur le Titanic dans le simulateur}
            \includegraphics[scale=0.5]{assets/Isaac_vs_Titanic.jpg}
            \label{fig:titanicForces}
        \end{center}
    \end{figure}

    \section{Création d'un pilote automatique pour le paquebot}

    \subsection{Objectifs et besoins}

    \subsection{Modélisation du pilote automatique en logique floue}

    \subsection{Réalisation du système floue à l'aide du \textit{framework}}

    \begin{figure}
        \begin{center}
            \caption{Code du pilote automatique pour l’interpréteur}
            \lstinputlisting[language=Bash, firstline=0, lastline=10, frame=single]{assets/automatic-pilot.fuzzy}
            \label{fig:codeAutoPilot}
        \end{center}
    \end{figure}

\end{document}
